% Packages
\usepackage[utf8]{inputenc}
\usepackage{graphicx}
\usepackage{enumerate}
\usepackage[hidelinks]{hyperref}
\usepackage[spanish]{babel} 
\usepackage{listings}
\usepackage{tikz}
\usepackage{lettrine}
\usepackage[T1]{fontenc}
\usepackage{palatino} 
\usepackage{type1ec}
\usepackage{amsmath}
\usepackage{lmodern} 
\usepackage{type1cm}
\usepackage{cite}
\usepackage{amsthm}
\usepackage{lscape} 
\usepackage{caption}
\usepackage{upgreek} % para poner letras griegas sin cursiva
\usepackage{cancel} % para tachar
\usepackage{mathdots} % para el comando \iddots
\usepackage{mathrsfs} % para formato de letra
\usepackage{stackrel} % para el comando \stackbin
\usepackage{dsfont}
\usepackage{verbatim} %Para usar verbatiminput para el codigo fuente
\usepackage{color}


\usepackage{pgfgantt}

\usepackage{amsfonts}
\usepackage{amssymb}

\usepackage{multicol}



\usepackage[lastexercise]{exercise}

\usepackage{pdfpages}
\usepackage{multirow}	% Para poder unir filas en las tablas

\usepackage{colortbl}	% Para colorear tablas
\usepackage{pifont}
\usepackage{ulem}


\usepackage[toc,page]{appendix}

\usepackage{fancybox,framed}
\usepackage[a4paper, total={6in, 8in}]{geometry}

\usepackage{float}
\usepackage{minted}

 \usepackage[version=4]{mhchem}


\usepackage{ifthen}
\newboolean{firstanswerofthechapter}  

\usepackage{xcolor}
\colorlet{lightcyan}{cyan!40!white}

\usepackage{stackengine}


\usepackage{xcolor}

\definecolor{codegreen}{rgb}{0,0.6,0}
\definecolor{codegray}{rgb}{0.5,0.5,0.5}
\definecolor{codepurple}{rgb}{0.58,0,0.82}
\definecolor{backcolour}{rgb}{0.95,0.95,0.92}

\lstdefinestyle{mystyle}{
    backgroundcolor=\color{backcolour},   
    commentstyle=\color{codegreen},
    keywordstyle=\color{magenta},
    numberstyle=\tiny\color{codegray},
    stringstyle=\color{codepurple},
    basicstyle=\ttfamily\footnotesize,
    breakatwhitespace=false,         
    breaklines=true,                 
    captionpos=b,                    
    keepspaces=true,                 
    numbers=left,                    
    numbersep=5pt,                  
    showspaces=false,                
    showstringspaces=false,
    showtabs=false,                  
    tabsize=2
}

\lstset{style=mystyle}





% Si quieres los capítulos con un rectangulo bordeado:
% \renewcommand{\familydefault}{\sfdefault}
% \usepackage{fancyhdr} % Custom headers and footers
% \pagestyle{fancyplain} % Makes all pages in the document conform to the custom headers and footers
% \fancyhead[L]{} % Empty left header
% \fancyhead[C]{} % Empty center header
% \fancyhead[R]{} % My name
% \fancyfoot[L]{} % Empty left footer
% \fancyfoot[C]{} % Empty center footer
% \fancyfoot[R]{\thepage} % Page numbering for right footer
% %\renewcommand{\headrulewidth}{0pt} % Remove header underlines
% \renewcommand{\footrulewidth}{0pt} % Remove footer underlines
% \setlength{\headheight}{13.6pt} % Customize the height of the header

% \usepackage{titlesec, blindtext, color}
% \definecolor{gray75}{gray}{0.75}
% \newcommand{\hsp}{\hspace{20pt}}
% \titleformat{\chapter}[hang]{\Huge\bfseries}{\thechapter\hsp\textcolor{gray75}{|}\hsp}{0pt}{\Huge\bfseries}
% \setcounter{secnumdepth}{4}
% \usepackage[Lenny]{fncychap}

\usepackage{booktabs, multirow} % for borders and merged ranges
\usepackage{soul}% for underlines
 % for cell colors
\usepackage{changepage,threeparttable} % for wide tables