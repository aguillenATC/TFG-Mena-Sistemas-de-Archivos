\cleardoublepage
\clearpage{}

%[Lo que va en el índice]{Lo que va en el documento}
\chapter[Metodología de trabajo]{Metodología de trabajo}
\section{Metodología de trabajo seguida en el documento}
Desde el comienzo del trabajo se ha seguido una especie de metodología ágil. Cada semana o cada dos semanas se celebraba una reunión con el tutor, estas reuniones se por norma general tenían lugar a través de Google Meet. En dichas reuniones exponía el trabajo que había realizado tras la reunión anterior, se resolvían las dudas que habían ido surgido y se proponían una serie de objetivos que, deseablemente había que trabajar de cara a la siguiente reunión. Gracias a seguir esta forma de organización, logramos avanzar en el proyecto, que a principio parecía inabarcable pero no resultó ser así. \\

Con el fin de tener un registro de como iba avanzando la memoria, creamos un repositorio en \href{http://www.github.com}{\textit{Github}}. Este repositorio estaba enlazado con el proyecto alojado en \href{http://www.overleaf.com}{\textit{Overleaf}} y al final de casa sesión de trabajo se subían los cambios. Desafortunadamente esta idea surgió tarde y las primeras versiones del documento no se encuentran registradas en el repositorio.\\

En referencia a la búsqueda de información el objetivo era primar la información de calidad, para ello siempre hemos tratado de acudir a documentaciones oficiales (si estaban disponibles) o directamente consultar bibliografía de prestigio. En caso de que esto no fuera posible, el siguiente paso era buscar información de empresas relacionadas con la cuestión a tratar (por ejemplo: Ext4 la documentación oficial disponible es muy escueta, mientras que la información disponible en la página de Fedora es bastante completa.) y por último lugar buscar en \textit{papers} ya que en algunos casos hay información muy buena, pero es complicado sacar información de este tipo de documentos por los motivos que se explicaron en la sección \ref{estado}.

\section{Metodología de los experimentos}

