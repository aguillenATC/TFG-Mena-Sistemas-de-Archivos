\cleardoublepage
\clearpage{}

%[Lo que va en el índice]{Lo que va en el documento}
\chapter[Metodología de trabajo]{Metodología de trabajo}
\section{Metodología de trabajo seguida en el documento}
Desde el comienzo del trabajo se ha seguido una especie de metodología ágil. Cada semana o cada dos semanas se celebraba una reunión con el tutor, estas reuniones se por norma general tenían lugar a través de Google Meet. En dichas reuniones exponía el trabajo que había realizado tras la reunión anterior, se resolvían las dudas que habían ido surgido y se proponían una serie de objetivos que, deseablemente había que trabajar de cara a la siguiente reunión. Gracias a seguir esta forma de organización, logramos avanzar en el proyecto, que a principio parecía inabarcable pero no resultó ser así. \\

Con el fin de tener un registro de como iba avanzando la memoria, creamos un repositorio en \href{http://www.github.com}{\textit{Github}}. Este repositorio estaba enlazado con el proyecto alojado en \href{http://www.overleaf.com}{\textit{Overleaf}} y al final de casa sesión de trabajo se subían los cambios. Desafortunadamente esta idea surgió tarde y las primeras versiones del documento no se encuentran registradas en el repositorio.\\

En referencia a la búsqueda de información el objetivo era primar la información de calidad, para ello siempre hemos tratado de acudir a documentaciones oficiales (si estaban disponibles) o directamente consultar bibliografía de prestigio. En caso de que esto no fuera posible, el siguiente paso era buscar información de empresas relacionadas con la cuestión a tratar (por ejemplo: Ext4 la documentación oficial disponible es muy escueta, mientras que la información disponible en la página de Fedora es bastante completa.) y por último lugar buscar en \textit{papers} ya que en algunos casos hay información muy buena, pero es complicado sacar información de este tipo de documentos por los motivos que se explicaron en la sección \ref{estado}.

\section{Metodología de los experimentos}
El objetivo de este trabajo requiere comparar el rendimiento de Btrfs, Ext4 y XFS en un único disco. Para realizar la comparativa de rendimiento, uno debería plantearse el uso de distintas herramientas de \textit{benchmarking} que sean capaces de mostrar el comportamiento de los sistemas de archivos bajo una determinada carga. Para este caso específico diremos que existen dos tipos de \textit{benchmarks}. \\

Por un lado tenemos aplicaciones reales (por ejemplo medir el rendimiento de una tarjeta gráfica utilizando un videojuego). El uso de este tipo de \textit{benchmarks} tiene aspectos positivos, pero resulta difícil encontrar aplicaciones suficientemente representativas que englobe el funcionamiento de un sistema de archivos en la mayoría de casos.\\

Por otro lado tenemos los \textit{benchmarks} sintéticos. Son programas artificiales que no realizan ningún trabajo real y útil. Las operaciones realizadas por estos \textit{benchmarks} se eligen cuidadosamente para que coincidan con las operaciones que realizaría otro tipo de programas. El objetivo es que las operaciones que realiza el \textit{benchmark} sean lo mas similares posible para así poder exportar los resultados al mundo real \cite{lilja_2000}.\\

La combinación de ambos tipos de \textit{benchmarks} para la medición de rendimiento de E/S del sistema de archivos producirá un resultado más representativo en lugar de depender únicamente de uno de los dos tipos de herramientas. Este proyecto hace uso tanto de \textit{benchmarks} sintéticos elaborados con \textit{Filebench} como de aplicaciones reales para las medidas.

\section{Máquina de pruebas}
Los experimentos fueron ejecutados en una máquina convencional con un procesador AMD con una frecuencia de reloj de 3.5 GHz. Ubuntu 16.04.7 fue la versión del sistema operativo en el que se ejecutaron las pruebas. El sistema tenía 2 discos, en un disco SSD se encontraba el sistema operativo y las herramientas de \textit{benchmark} y el disco HDD se utilizaba exclusivamente para el montaje de los sistemas de archivos. La tabla siguiente muestra todos los detalles sobre el hardware y software utilizado.
\begin{table}
    \centering
    \begin{tabular}{|l|l|}
    \hline
        Componente & Modelo \\ \hline\hline
        CPU & AMD Ryzen 3 1300X @ 3.5 GHz \\ \hline
        Memoria & Crucial CT8G4DFS824A. 8 GB DDR4 2400 MHz CL17 \\ \hline
        Placa Base & ASUS EX-A320M \\ \hline
        SSD del Sistema & Kingston A400 SSD SATA3 \\ \hline
        HDD de Pruebas & Seagate Barracuda ST500DM002-1BD14 500GB \\ \hline
    \end{tabular}
    \caption{Especificaciones Hardware}
\label{table:1}
\end{table}

\begin{table}
    \centering
    \begin{tabular}{|l|l|}
    \hline
        Software & Versión \\ \hline\hline
        Ubuntu 16.04.7 & Kernel XXXXXXXXXXXXXXXXXXXX \\ \hline
        Filebench & 1.5-alpha3 \\ \hline
    \end{tabular}
    \caption{Especificaciones Software}
\label{table:2}
\end{table}

\section{Herramientas de \textit{benchmarking}}
\subsection{Iozone}
El software Iozone en general es utilizado para medir el rendimiento del sistema de archivos, utilizando distintos tipos de pruebas. Utilizamos esta herramienta para medir el rendimiento en operaciones de lectura, escritura, re-lectura, re-escritura, lectura aleatoria y escritura aleatoria. Se han seleccionado estas operaciones debido a que son operaciones que son ejecutadas por cualquier aplicación, utilizando cualquier tipo de sistema de archivos. Iozone es una suite ampliamente extendida en el mundo de los benchmarks, probablemente debido a su variedad de ajustes y es por ello por lo que la vamos a utilizar.
\subsection{Filebench}
Utilizaremos esta herramienta principalmente por disponer de un lenguaje para modelar carga. Es decir, utilizaremos el lenguaje de modelación de carga de Filebench para crear benchmarks sintéticos. El objetivo es crear un benchmark que se simule un uso real del disco por parte de una aplicación genérica o un usuario genérico. 
\section{Entorno de pruebas y repetición de experimentos}
\section{Instalación de paquetes}
\section{Test Anova}



