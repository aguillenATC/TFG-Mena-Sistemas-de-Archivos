

\cleardoublepage
\clearpage{}

%[Lo que va en el índice]{Lo que va en el documento}
\chapter[Presupuesto]{Presupuesto
}
Teniendo en cuenta el diagrama de Gantt \ref{fig:gantt-chart-post}, donde se muestra la duración de las diferentes tareas, podemos estimar el coste real del desarrollo. El desarrollo del proyecto se ha llevado a cabo de manera continua. El proyecto ha durado aproximadamente 22 semanas, de media se le han dedicado una 15 horas por semana, es decir 3 horas al día (sin contar fin de semana) por lo que en total se le ha dedicado aproximadamente 330 horas al proyecto.\\

Según la web \textit{Glassdoor} el sueldo medio de un ingeniero junior en España al año es de 24.887€ \cite{glassdoor}. Si partimos de la base de que este sueldo se divide en 14 pagas, obtenemos que el sueldo mensual es de 1777,64€ \textcolor{blue}{¿NETO/BRUTO?}. Si se trabajan 8 horas diarias, 5 días por semana se obtiene que el sueldo por hora es de 11,11€/h. También se deben contabilizar en el precio del proyecto, las horas dedicadas por los tutores puesto que en condiciones reales, se hubiese tenido que consultar con un especialista. El salario por hora de un profesor titular universitario en 2019 según el portal de transparencia de la Universidad de Sevilla \cite{sueldo-US} es de 15,26€/h y se habrán invertido unas 20 horas en tutorías. El precio del proyecto aproximadamente se detalla en la siguiente tabla.


\begin{table}[H]\centering
\scriptsize
\begin{tabular}{lrrrr}\toprule
Recurso &Precio/hora &Horas totales &Precio total \\\midrule
Ingeniero Junior &11.11000€ &330 &3666.3€ \\
Profesional &15.26000€ &20 &305.2€ \\
& & &3971.5€\\
\bottomrule
\end{tabular}
\caption{Tabla de costo del proyecto detallado.}
\end{table}



