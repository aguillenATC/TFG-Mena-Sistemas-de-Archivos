

\cleardoublepage
\clearpage{}

%[Lo que va en el índice]{Lo que va en el documento}
\chapter[Planificación]{Planificación}
\section{Previa}

\begin{figure}[H]
    \centering
    \scalebox{0.60}{
\begin{ganttchart}[%Specs
     y unit title=0.5cm,
     y unit chart=0.7cm,
     vgrid,hgrid,
     title height=1,
     title label font=\bfseries\footnotesize,
     bar/.style={fill=blue},
     bar height=0.4,
     group right shift=0,
     group top shift=0.7,
     group height=.3,
     group peaks width={0.2},
     inline]{1}{36}
    %labels
    \gantttitle{Nov}{4} %1-4
    \gantttitle{Dic}{4} 
    \gantttitle{Ene}{4}
    \gantttitle{Feb}{4}
    \gantttitle{Mar}{4}
    \gantttitle{Abr}{4}
    \gantttitle{May}{4}
    \gantttitle{Jun}{4}
    \gantttitle{Jul}{4}\\
    % Setting group if any
    \ganttgroup[inline=false]{Estudio}{3}{7}\\
    \ganttbar[inline=false, bar/.style={fill=red}]{Planificación}{3}{4}\\
    \ganttbar[inline=false, bar/.style={fill=red}]{Investigación sobre el tema}{5}{7}\\
    \ganttbar[inline=false, bar/.style={fill=red}]{Busqueda de trabajos similares}{4}{7}\\

    
    \ganttgroup[inline=false]{Documentación}{8}{32}\\
    \ganttbar[inline=false, bar/.style={fill=cyan}]{Introducción}{8}{9}\\
    \ganttbar[inline=false, bar/.style={fill=cyan}]{Fundementos y Estado del Arte}{9}{11}\\
    \ganttbar[inline=false, bar/.style={fill=cyan}]{Metodología}{12}{14}\\
    \ganttbar[inline=false, bar/.style={fill=cyan}]{Experimentación}{15}{27}\\
    \ganttbar[inline=false, bar/.style={fill=cyan}]{Conclusión, planificacion y presupuesto}{28}{29}\\
    \ganttbar[inline=false, bar/.style={fill=cyan}]{Revisión}{30}{32}\\
    
    \ganttgroup[inline=false]{Experimentación}{15}{23}\\
    \ganttbar[inline=false, bar/.style={fill=purple}]{Instalación de herramientas}{15}{15}\\
    \ganttbar[inline=false, bar/.style={fill=purple}]{Ejecución test KnowSeq}{16}{17}\\
    \ganttbar[inline=false, bar/.style={fill=purple}]{Ejecución test Iozone}{19}{20}\\
    \ganttbar[inline=false, bar/.style={fill=purple}]{Creación del test y ejecución de Filebench}{22}{23}\\
    
      \ganttgroup[inline=false]{Preparación de datos}{18}{27}\\
    \ganttbar[inline=false, bar/.style={fill=orange}]{Creación de tablas de datos de KnowSeq}{18}{18}\\
    \ganttbar[inline=false, bar/.style={fill=orange}]{Creación de tablas de datos de Iozone}{21}{21}\\
    \ganttbar[inline=false, bar/.style={fill=orange}]{Creación de tablas de datos de  Filebench}{24}{24}\\
    \ganttbar[inline=false, bar/.style={fill=orange}]{Realización de Tests Anova}{25}{27}\\
    
    
\end{ganttchart}}
    \caption{Diagrama de Gantt que muestra la temporización esperada del proyecto.}
    \label{fig:gantt-chart-prev}
\end{figure}
\section{Posterior}
\begin{figure}[H]
    \centering
    \scalebox{0.60}{
\begin{ganttchart}[%Specs
     y unit title=0.5cm,
     y unit chart=0.7cm,
     vgrid,hgrid,
     title height=1,
     title label font=\bfseries\footnotesize,
     bar/.style={fill=blue},
     bar height=0.4,
     group right shift=0,
     group top shift=0.7,
     group height=.3,
     group peaks width={0.2},
     inline]{1}{36}
    %labels
    \gantttitle{Nov}{4} %1-4
    \gantttitle{Dic}{4} 
    \gantttitle{Ene}{4}
    \gantttitle{Feb}{4}
    \gantttitle{Mar}{4}
    \gantttitle{Abr}{4}
    \gantttitle{May}{4}
    \gantttitle{Jun}{4}
    \gantttitle{Jul}{4}\\
    % Setting group if any
    \ganttgroup[inline=false]{Estudio}{3}{7}\\
    \ganttbar[inline=false, bar/.style={fill=red}]{Planificación}{3}{4}\\
    \ganttbar[inline=false, bar/.style={fill=red}]{Investigación sobre el tema}{5}{7}\\
    \ganttbar[inline=false, bar/.style={fill=red}]{Busqueda de trabajos similares}{4}{7}\\

    
    \ganttgroup[inline=false]{Documentación}{12}{33}\\
    \ganttbar[inline=false, bar/.style={fill=cyan}]{Introducción}{12}{16}\\
    \ganttbar[inline=false, bar/.style={fill=cyan}]{Fundementos y Estado del Arte}{17}{20}\\
    \ganttbar[inline=false, bar/.style={fill=cyan}]{Metodología}{21}{25}\\
    \ganttbar[inline=false, bar/.style={fill=cyan}]{Experimentación}{26}{31}\\
    \ganttbar[inline=false, bar/.style={fill=cyan}]{Conclusión, planificacion y presupuesto}{32}{32}\\
    \ganttbar[inline=false, bar/.style={fill=cyan}]{Revisión}{32}{33}\\
    
    \ganttgroup[inline=false]{Experimentación}{23}{30}\\
    \ganttbar[inline=false, bar/.style={fill=purple}]{Instalación de herramientas y testeo}{23}{24}\\
    \ganttbar[inline=false, bar/.style={fill=purple}]{Ejecución test KnowSeq}{26}{27}\\
    \ganttbar[inline=false, bar/.style={fill=purple}]{Ejecución test Iozone}{28}{29}\\
    \ganttbar[inline=false, bar/.style={fill=purple}]{Creación del test y ejecución de Filebench}{30}{30}\\
    
      \ganttgroup[inline=false]{Preparación de datos}{27}{30}\\
    \ganttbar[inline=false, bar/.style={fill=orange}]{Creación de tablas de datos de KnowSeq}{27}{27}\\
    \ganttbar[inline=false, bar/.style={fill=orange}]{Creación de tablas de datos de Iozone}{29}{29}\\
    \ganttbar[inline=false, bar/.style={fill=orange}]{Creación de tablas de datos de  Filebench}{30}{30}\\
    \ganttbar[inline=false, bar/.style={fill=orange}]{Realización de Tests Anova}{27}{30}\\
    
    
\end{ganttchart}}
    \caption{Diagrama de Gantt que muestra la temporización esperada del proyecto.}
    \label{fig:gantt-chart-post}
\end{figure}


En la planificación inicial se pretendía no simultanear varias tareas a la vez, por ello las tareas de experimentación y preparación de datos están más dilatadas en el tiempo. Finalmente no fue posible seguir esta planificación, principalmente, por que el trabajo se empezó a desarrollar más tarde de lo esperado. Conllevó que la planificación finalmente fuese  muy distinta, teniendo que simultanear las tareas de documentación, experimentación y preparación de datos. El retraso trajo consigo la disminución general de tiempo para todas las tareas


